\documentclass[a4paper, 11pt]{report}

\usepackage{color}
\usepackage[utf8]{inputenc}
\usepackage[frenchb]{babel}
\usepackage{graphicx}
%\usepackage[cm]{fullpage}

\graphicspath{{./img/}}

\def\maketitle{
  \hspace{-3.5cm}
    \hbox to 0px{\includegraphics[scale = 0.7]{LOGO_UNS.png}}  
    \hbox to 12cm{}  
    \hbox to 0px{\includegraphics[scale = 1.1]{Polytech.jpg}} 
  \vfill
  \begin{center}\leavevmode
    \normalfont
    {\LARGE \textbf{CAMASH: DOCUMENT DESCRIPTIF DU JEU}\par}%
    ~~\\
    ~~\\
    {\Large Auteurs:\textbf{ Jérémy~Gabriele, Clément~Léger, Romaric~Pighetti} \par}%
    ~~\\
    ~~\\
    {\Large Encadrant: \textbf{Jean-Paul Stromboni} \par}%
    ~~\\
    ~~\\
    {\Large \'Ecole: \textbf{Polytech'Nice-Sophia} \par}%
    ~~\\
    ~~\\
    {\Large \textbf{\today}   \par}%
    ~~\\
    ~~\\
    {\Large \textbf{Version 0.1}   \par}
    \vspace {3cm}
    {\Large \textbf{Résumé:}\par}%
    \end{center}%
    Ce document présente et décrit le jeu qui sera développer par Jérémy~Gabriele, Clément~Léger et 
    Romaric~Pighetti pour leur projet dans le cadre du cours de Conception d'Application Multimédia
    Animée en Situation de Handicap (CAMASH). Le jeu développé au cours de ce projet est un jeu de 
    parcours de donjon développé à l'aide de javascript, html5 et css. Ce document décrit tous les éléments
    du jeu ainsi que les différentes interactions possible. Il donne aussi un planning prévisionnel du 
    développement de l'application.
  \vfill
  \null
  \clearpage
}

\begin{document}
  \maketitle
  
  \chapter*{Introduction}
    Le document descriptif du jeu a pour but de présenter le projet à l'équipe encadrante, afin de montrer
    l'état de nos réflexions et nos idées, ainsi qu'un planning prévisionnel de conception pour qu'elle puisse
    juger de la faisabilité du projet. Mais ce document a aussi pour objectif de guider les développeurs tout
    au long de la conception du jeu. C'est pourquoi ce document comporte des informations très détaillées
    sur les différents points du jeu. Lorsque quelqu'un a un doute sur le comportement que doit avoir le jeu
    dans une certaine situation ou ce qui doit \^etre affiché, il doit pouvoir trouver la réponse dans ce
    document. Nous trouverons donc dans une première partie une présentation générale du jeu qui sera
    développé. Nous rentrerons ensuite dans les détails du gameplay et des mécanismes de jeu. Les détails
    sur l'interface graphique, comportant les diagrammes d'écran, suivra avant de se pencher sur certain 
    détails techniques tel que l'intelligence artificielle ou les plate formes visées. Enfin un planning 
    prévisionnel sera donné, permettant de voir l'organisation prévu par le groupe pour que le 
    développement soit terminé dans les temps.

  \tableofcontents

  \part{Présentation du Jeu}
    \chapter{Concept du jeu}
      Le jeu présenté ici est un jeu de parcours de donjon. Le joueur commence dans une pièce et doit 
      parcourir le donjon, formé de plusieurs pièces, afin de trouver un objet. Cependant, les pièces du
      donjon ne sont pas éclairées. Le joueur possède donc une source de lumière lui permettant d'éclairer
      une certaine zone autour de lui. Mais cette source de lumière ne peut l'éclairer indéfiniment, il devra 
      trouver du carburant pour l'alimenter régulièrement au cours de son parcours s'il veut conserver de 
      la lumière. Le donjon est de plus infesté de monstres, et le joueur sera incapable de se battre tant qu'il
      tiendra sa source de lumière à la main. Il devra donc la déposer pour pouvoir se battre et la reprendre
      ensuite. Notre héros possède aussi un sonar, outil lui permettant d'envoyer une onde et d'en observer
      les échos pour deviner les objets l'entourant. 

    \chapter{Ensemble des fonctions}

    \chapter{Genre}
  
    \chapter{Public visé}

    \chapter{Aspect général}

    \chapter{Portée du projet}

  \part{Gameplay et Mécanismes de Jeu}

    \chapter{Environnement}
      Dans ce chapitre nous allons nous efforcer de décrire le plus finement possible l'environnement dans
      lequel nos acteurs vont évoluer. Nous allons pour cela décrire les composants pouvant apara\^itre 
      puis nous nous pencherons sur les lois physiques qui régissent notre monde virtuel.

      \section{Le monde}
        Notre monde est simple, nous naviguons dans un donjon qui est composé de salles. Chaque salle 
        pouvant avoir des murs, des portes, des objets posés au sol ainsi que des ennemis s'y trouvant.
        Décrivons plus en détail ces éléments.

        \subsection{Le héros}
          Le héros est un simple \^etre humain. Il parcours les salles du donjon à la recherche du trésor.

        \subsection{Les monstres}
          Les monstres sont des entités malfaisantes qui sont présentes dans les salles et qui ne peuvent
          changer de salle. Ils attaqueront le héros si celui-ci s'approche trop.

        \subsection{Les objets}
          Les objets sont décrits plus amplement dans un chapitre suivant, ici seul nous intéresse le fait qu'ils
          sont déposés sur le sol dans les salles et n'emp\^echent pas les acteurs de se déplacer sur ces 
          cases. Ils peuvent aussi \^etre portés par des monstres, ils tomberont alors au sol à l'endroit où le
          monstre a été tué. Les limites sur le port d'objet par le joueur sont explicités plus loin dans cette 
          partie.
          
        \subsection{Les objets massifs}
          Les objets massifs sont des objets que le héros ne peut pas ramasser et sur lesquels on ne peut se
          déplacer. Ce sont des éléments de décor comme des tables, des chaises, des coffres ou autre.
          
        \subsection{Les murs}
          Les murs sont des zones infranchissable dans une salle. Ils composent aussi le tour de la salle là où
          il n'y a pas de porte.
          
        \subsection{Les portes}
          Les portes sont présentes seulement sur le bord des salles et permettent au héros de passer aux 
          salles adjacentes.
          
        \subsection{Les salles}
          Une salle est définit par un ensemble de murs, des objets massifs (tables, chaises), des monstres
          présents dans la salle et des portes permettant de passer à d'autres salles.
          
        \subsection{Le donjon}
          Le donjon est l'ensemble des salles dans lesquels le joueur évolue au cours d'une partie. Chaque
          nouvelle partie donne naissance à un nouveau donjon généré au début de celle-ci. 
          Le donjon est représenté par un graphe de
          salles, définissant ainsi les liaisons entre les salles, sachant qu'une salle ne peut   \^etre connectée
          à plus de 4 autres salles. Le joueur n'a pas accès à une représentation du donjon. Il pourra créer sa
          propre carte s'il le souhaite.

      \section{La physique}
        
        \subsection{Positions autorisées pour les objets et les acteurs}
          Les objets et les acteurs ne peuvent se trouver que sur des cases correspondant au sol d'une salle, 
          ils ne peuvent se trouver sur les murs ou les objets dit massifs. Deux acteurs ne peuvent occuper
          une m\^eme position au m\^eme moment.
        
        \subsection{La lumière}
          La lumière ne peut pas traverser les murs. Si une source de lumière est placée proche d'un mur, de
          telle sorte que la zone qu'elle peut éclairer s'étant au delà du mur, l'éclairage s'arrête au niveau
          du mur. La lumière traverse les objets massifs, ce qui permet de les éclairer et d'éclairer au delà de
          ceux-ci. De même la présence de monstre n'arrête pas la lumière.
          
        \subsection{Les ondes du sonar}
          Les ondes du sonar ne traversent rien. Dès qu'un objet est rencontré par l'onde du sonar (mur, objet,
          objet massif, monstre), celle-ci est renvoyée dans la direction de laquelle elle provenait.
          
        %\subsection{Les projectiles}
          %Les projectiles des armes (des monstres ou du héros) passent au dessus des objets et des objets 
          %massifs mais ne peuvent pas traverser les murs. Chaque projectile est muni d'une distance 
          %maximum qu'il peut parcourir, au delà de cette distance il disparaît sans effectuer de dégâts.
          %Les projectiles disparaissent aussi lorsqu'ils entrent en contact avec un autre acteur (monstre ou 
          %héros) en lui infligeant les dégâts associés au projectile en question.

    \chapter{Objectifs}
      \subsection{Objectif principal}
        Trouver la clé pour pouvoir s'échapper du donjon.
        
      \subsection{Objectif secondaire}
        Trouver les bairks et les emmener hors du donjon.
      
    \chapter{Objets}
      \section{Objets massifs}
        Ces objets sont simplement déposé sur le sol dans les salles. Aucun acteur ne peut se déplacer
        à un endroit occupé par un objet de cette catégorie. En voici la liste:
        \begin{itemize}
        \item{Table}
        \item{Chaise}
        \item{Coffre}
        \end{itemize}
        
      \section{Objets récupérable}
        Ces objets sont ramassé dès que le héros passe sur la case qu'ils occupent.
        
        \subsection{Cœur de vie}
          Il redonne un cœur de vie au héros. Si la vie du héros est pleine, le cœur disparaît sans avoir d'effet.
          Si il manque moins d'un cœur de vie au joueur, sa vie est mise au maximum et le cœur de vie
          disparaît.
        
        \subsection{Clé}
          Le clé est stockée dans dans l'inventaire du héros lorsqu'il la ramasse. Elle n'existe qu'en un 
          exemplaire dans un donjon et permet au héros de sortir du donjon.
          
        \subsection{Lanterne}
          La lanterne peut-être déposée au sol. Lorsqu'elle est sur le sol elle est considérée comme un objet
          récupérable. Le héros la reprend donc automatiquement lorsqu'il passe à l'endroit où elle se trouve.
          
        \subsection{Huile pour lanterne}
          L'huile pour lanterne permet de remplir le réservoir de la lanterne afin qu'elle ne s'éteigne pas. Elle
          permet de régénérer 1/3 du temps total d'éclairage de la lanterne. Si le réservoir de la lanterne est 
          plein, l'huile n'a aucun effet mais disparaît tout de même. Si moins d'un tiers du réservoir est vidé, 
          le réservoir est remplis et l'huile disparaît.
          
        \subsection{Bairks}
          Bien que se déplaçant, les bairks peuvent être considérés comme des objets récupérable. Lorsque
          le joueur passe à l'endroit où un bairks se trouve, il le prend avec lui automatiquement. Le bairks
          sera dès lors caché dans une poche du héros.
          
    \chapter{Actions}
      
      \section{Déplacements}
        Le joueur peut déplacer le héros vers le haut, vers le bas, vers la droite ou vers la gauche. Si la 
        direction vers laquelle le joueur veut déplacer le héros est occupée par un mur ou un objet massif, 
        le héros ne se déplacera pas.

      \section{Poser la lampe}
        Le héros peut aussi déposer sa lanterne. Cette action a pour effet de poser la lanterne dans la zone
        occupée par le héros au moment de l'action.
        
      \section{Ramasser un objet}
        Le héros ramassera automatiquement et utilisera immédiatement (si approprié) les objet récupérable
        occupant les espaces qu'il traverse. Si ces objets ne sont pas utilisable immédiatement ils seront
        laissé sur place ou stocké selon les cas. Le paragraphe destiné aux objets décrit le comportement
        désiré pour chaque objet.
        
      \section{Attaquer}
        On peut aussi faire attaquer le héros à l'aide de son couteau. Le héros n'est pas capable d'utiliser son
        couteau s'il porte la lanterne, il doit donc la poser avant de pouvoir attaquer. Si la lanterne n'est pas
        posée et qu'une action d'attaque est demandée, rien ne se passe.
        
      \section{Utilisation du sonar}
        On peut déclencher le sonar que possède le héros. Cela aura pour effet de lancer une onde sonar de
        façon circulaire autour du personnage. La propagation de cette onde sera visible sur l'écran de jeu 
        ainsi que ces réflexions, permettant au joueur de se repérer quelque peut. Après avoir été utilisé, le 
        sonar est désactivé pendant quelques secondes. Si le joueur demande à réutiliser le sonar dans cette
        période de temps, rien ne se passe.

  \part{Interface}
    \chapter{Interface graphique}
      \section{Diagramme d'écran}

      \section{Description des écrans}
        \subsection{Menu principal}
        
        \subsection{etc.}

    \chapter{Sons}

    \chapter{Contrôle}

  \part{Technique}
    
    \chapter{Intelligence artificielle}
      
      \section{Les monstres}
        L'intelligence des monstres et très sommaire, ils se déplacent aléatoirement dans la salle qu'ils 
        occupent. Si le héros se trouve assez près d'eux, ils se dirigent vers lui et commence à l'attaquer.
        On peut tout de m\^eme distinguer différents comportements vis-à-vis de la lumière:
        \begin{itemize}
        \item{Ceux qui fuient la lumière}
        \item{Ceux qui sont attiré par la lumière}
        \item{Ceux qui sont indifférent à la lumière}
        \end{itemize}
        Le comportement des ennemies qui sont indifférent à la lumière auront exactement le comportement
        décrit ci-dessus. Les monstres fuyant la lumière ne rentrerons pas dans l'éclairage fournis par la 
        lanterne. Ils resteront donc en limite de la zone éclairé et suivront le héros de cette manière si celui-
        ci s'approche de trop près. Enfin les ennemies attirés par la lumière chercheront à tout pris à aller 
        dans la zone éclairée par la lanterne.
        
      \section{Les bairks}
        Les bairks se déplacent aléatoirement dans les salles qu'ils occupent tout en fuyant la lumière. Un
        bairks ne s'aventurera jamais sur une case éclairée.

    \chapter{Plate forme ciblée}
      Le jeu devra s'exécuter correctement dans les navigateurs suivant:
      \begin{itemize}
      \item{Firefox >4}
      \item{Chrome >X}
      
      \end{itemize}

    \chapter{Langages utilisés}

    \chapter{Cadre d'application et bibliothèques utilisées}

  \part{Planning et organisation}

    \chapter{Planning}
      
    \chapter{Répartition des t\^aches}

  \chapter*{Conclusion} 
    Le jeu décrit dans ce document est un jeu assez ambitieux pour le temps accordé au développement. 
    On peut cependant espérer arriver à une première version contenant les éléments principaux et 
    permettant déjà de jouer, m\^eme si des améliorations de contenu seront certainement nécessaires
    pour que l'expérience de jeu ne soit pas trop monotone. De m\^eme, le comportement des monstres
    et la génération des salles pourront \^etre améliorer dans le futur si ce jeu est prometteur. La 
    réalisation de se projet doit pouvoir donner une première version jouable du jeu intégrant la prise en
    compte d'un handicap auditif et implémentant toutes les fonctionnalités présentées. Du contenu et
    des améliorations pouvant \^etre apportées ensuite.
   
  \clearpage

\end{document}
